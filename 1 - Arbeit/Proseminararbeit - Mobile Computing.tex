% Auf der Basis von dem Springer llncs Style
\documentclass{llncs}					% Springer Style
\usepackage{llncsdoc}					% Springer Dokument
\usepackage[utf8]{inputenc}				% Umlaute, Sonderzeichen
\usepackage[ngerman]{babel}				% deutsche Sprache
\usepackage{enumitem}					% Listen
\usepackage{graphicx}					% Grafiken
\usepackage{hyperref}					% Hyperlinks
\usepackage[nonumberlist]{glossaries}	% Glossar
\usepackage{amsmath}					% Mathematik

\bibliographystyle{unsrt}

\makenoidxglossaries

\newglossaryentry{1}{
	name=A Haptic Back Display for Attentional and Directional Cueing,
	description={Der Raspberry Pi ist ein Einplatinencomputer. In diesem Projekt dient der Raspberry Pi als Hardwareplattform, um Messwerte aus angeschlossenen Sensoren auszulesen}
}

\title{Aufmerksamkeitssteuerung durch Haptische Schnittstellen in Überwachungstätigkeiten}
\author{Leon Huck\thanks{Unter der Betreuung von: Erik Pescara}}
\institute{Karlsruher Institut für Technology}
\date{17.06.2019}

\begin{document}
	
%Titel der Arbeit
\maketitle

%Schlagworte der Arbeit
\begin{description}
	\item ToDo
\end{description}

%Abstrakt der Arbeit
\begin{abstract}
	ToDo
\end{abstract}

\begin{flushleft}
	Forschungsfrage:
	Wo werden Haptische Schnittstellen bereits heute zur Aufmerksamkeitssteuerung, bei Beobachtugnsaufgaben, eingesetzt und wie könnte man diese Bereiche erweitern?
	
	Um diese Frage beantworten zu können muss ich zuerst:
	\item Klären, was Haptische Schnittstellen sind. Welche Möglichkeiten zur  Entwicklung und Anpassung es gibt. Welche Probleme sie gemeinsam haben
	\item Wie Aufmerksamkeit, zumindest auf einem Abstrakten Niveau, zustande kommt. Wieso eine Beeinflussung durch haptische Schnittstellen sinnvoll ist. ?Welche Probleme auftreten?
	\item Was mit Überwachungstätigkeit gemeint ist. Welche besonderen Aspekte zu berücksichtigen sind.
	\item Anhand dieses Rahmens kann ich dann sinnvolle Bereiche auswählen und zusammenführen.
\end{flushleft}
%Inhaltsverzeichnis
%ToDo: Überdenken, ob nicht entfernen
\newpage
\tableofcontents
\newpage

%Hinführung zu der Arbeit
\newpage
\section{Einleitung}
Fragen/Teilgebiete/Gliederungspunkte/Absätze:
Wovon handelt die Arbeit?
Was ist ihr Ziel?
Welche Erkenntisse sind zu finden?
Wie kann ich zu dem Thema hinführen?

%Beschreibung der Teilgebiete, die im späteren Verlauf zusammengeführt werden
\section{Die Thematischen Teilgebiete}
Fragen/Teilgebiete/Gliederungspunkte/Absätze:
Warum ist die Unterteilung in diese Teilgebiete wichtig?
Wo grenzen sie sich ab?
Welche Gebiete wären sonst noch wichtig gewesen, werden aber wegen einem zu großen Umfang ausgelassen?
Wie sind diese Unterteilungen zu stande gekommen?

Die Arbeit setzt sich aus drei Teilen zusammen. Dabei ist das erste das Anfällt die Aufmerksamkeitssteuerung. Also wie bringe ich jemanden dazu dort hin zu schauen, wo die Aktion ist. Gefolgt von Haptischen Schnittstellen. Diese sollen im Unterschied zu Ton überwiegend über die Haut Informationen übertragen. Zulätz wird das ganze in den Rahmen einer Überwachungsaufgabe gefasst.

\subsection{Aufmerksamkeitssteuerung}
Fragen/Teilgebiete/Gliederungspunkte/Absätze:
Aufmerksamkeit ist ein weitläufiges Feld. Deshalb ist es für die Diskussion in der Arbeit wichtig genau zu definieren, welche Arten der Aufmerksamkeit behandelt werden.

\subsection{Haptische Schnittstellen}
Fragen/Teilgebiete/Gliederungspunkte/Absätze:
Warum ist es überhaupt möglich Haptische Schnittstellen für diese Funktion einzusetzen?
Welche Möglichkeiten gibt es Aufmerksamkeit zu lenken?
Wie Unterscheiden sich die Schnittstellen in:
•Mechanischen Konzepten
•Effektivität
•
Der Mensch verfügt über einen Tastsinn. Haptische Schnittstellen nutzen diesen Sinn aus, um Informationen zu übertragen.

Hierbei existieren unterschiedliche Möglichkeiten die Stimulationen zu erzeugen.
Der erste punkt, der angesprochen werden muss, ist welche Mechanik zugrundeliegt. Eine Möglichkeit ist elektromotoren zu verwenden um eine Platte von einer bestimmten größe und material in schwingung zu versetzen. Es werden auch Geräte, die Informationen über Stromimpulse vermitteln unter die haptik gerechnet. Denn sie alle agieren über das selbe Organ, die Haut. Jeder einzelne dieser Faktoren kann genutzt werden um unterschiedliche Eindrücke zu hinterlassen.

Will man besipielsweise einen grundlegenden Alarm simulieren, wäre ein eindringliches haptisches Signal zu wählen.

Um zu wissen, welche Möglichkeiten Systeme besitzen um Informationen darzustellen ist es wichtig alle Faktoren zu kennen. Cite1

„Übersicht von allen Faktoren erstellen“

„Exemplarisches Durchspielen eines Anwendungsfalls“

Weitere Anwendungen, die hier aufgelistet werden werden sich immer nach dem selben Schema zusammensetzen:
*Welche Information soll übermittelt werden
-hier ist besonders die Komplexität der Information wichtig, da man auch immer noch die menschlichen Limitationen berücksichtigen muss.
Bsp. 26 Tactoren einer für jeden Buchstaben == Computer würde dies Verstehen, mensch jedoch limitiert.

Der Mensch ist also immer eine zu berücksichtigende Komponente

*welche weiteren Einschränkungen gibt es.
Zum Beispiel kann man in vielen Fällen die Fingerspitzen nicht mit Tactoren ausstatten. Cite 4
Belastungsszenarien wie etwa militär cite...

Also ergeben sich allgemeine Anforderungen und schließlich noch erweiterte

Lösungen oft durch vereinfachung der datenübertragung. Also genaue Signale gegenüber komplexen, vermeindlich informationsreichen.

Auslagerung der komplexen Informationsaufnahme an andere Sinnesorgane. Verwende Haptik nur um Aufmerksamkeit zu gewinnen. So wie evolutionär entwickelt. Cite 4

\subsection{Überwachungsaufgaben}
Fragen/Teilgebiete/Gliederungspunkte/Absätze:

\section{Anwendungen}
Fragen/Teilgebiete/Gliederungspunkte/Absätze:

\subsection{Sinneswiederherstellung}
Fragen/Teilgebiete/Gliederungspunkte/Absätze:
Menschliche Sinne können, von Geburt an oder im laufe der Zeit, nicht, oder nur eingeschränkt, funktionsfähig sein. Um diesen Leistungsverlust ausgleichen zu können bedarf es technischer Hilfsmittel. Hierbei bietet die menschliche Haut eine Möglichkeit zur Kommunikation mit der Außenwelt. Im folgenden soll beschrieben werden, wie diese Eigenschaft genutzt werden kann um, über haptische Schnittstellen, ausgewählte Sinne wiederherstellen zu können.

\subsubsection{Sehvermögen}
Fragen/Teilgebiete/Gliederungspunkte/Absätze:
Nach dem Stand der aktuellen Forschung ist das Auge das Leistungsstärkste Sinnesorgan, gemessen an der übertragenen Datenmenge\cite{Koch2006}. Dabei liegt die absolute Leistung ca. bei der eines Ethernet-Kabels mit 10 Mbit/s\cite{Koch2006}. Der Sehsinn kann somit bereits aus technischen gründen nicht vollständig über die Haut simuliert werden.
\subsection{Kommunikation}
Fragen/Teilgebiete/Gliederungspunkte/Absätze:

\subsection{Leistungssteigerung}
Fragen/Teilgebiete/Gliederungspunkte/Absätze:

\subsection{Wahrnehmungsspektrum Erweiterung}
Fragen/Teilgebiete/Gliederungspunkte/Absätze:

\subsection{Zuverlässigkeit Erzeugung}
Fragen/Teilgebiete/Gliederungspunkte/Absätze:

\section{Zusammenfassung und Ausblick}
Fragen/Teilgebiete/Gliederungspunkte/Absätze:

%Zusätzliche Informationen und Formalitäten
\section{Anhang}

\clearpage
\subsection{Glossar}\label{glossar}

\renewcommand*{\glossarysection}[2][]{}	% prevents double glossary section heading
\printnoidxglossaries				% generate pdf twice when adding new entries

\subsection{Selbständigkeitserklärung}

\clearpage
\bibliography{Literaturreferenzen}

\end{document}
