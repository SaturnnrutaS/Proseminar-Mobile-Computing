% Auf der Basis von dem Springer llncs Style
\documentclass{llncs}					% Springer Style
\usepackage{llncsdoc}					% Springer Dokument
\usepackage[utf8]{inputenc}				% Umlaute, Sonderzeichen
\usepackage[ngerman]{babel}				% deutsche Sprache
\usepackage{enumitem}					% Listen
\usepackage{graphicx}					% Grafiken
\usepackage{hyperref}					% Hyperlinks
\usepackage[nonumberlist]{glossaries}	% Glossar
\usepackage{amsmath}					% Mathematik

\bibliographystyle{unsrt}

\makenoidxglossaries

\newglossaryentry{Aktuator}{
	name=Aktuator,
	plural=Aktuatoren,
	description={Bauelement, welches elektrische Signale in andere physikalische Größen, wie beispielsweise Bewegung, umsetzt.}
}

\title{Aufmerksamkeitssteuerung durch Haptische Schnittstellen in Überwachungstätigkeiten}
\author{Leon Huck\thanks{Unter der Betreuung von: Erik Pescara}}
\institute{Karlsruher Institut für Technologie}
\date{17.06.2019}

\begin{document}
	
%Titel der Arbeit
\maketitle

%Schlagworte der Arbeit
\begin{description}
	\item ToDo
\end{description}

%Abstrakt der Arbeit
\begin{abstract}
	ToDo
\end{abstract}

\begin{flushleft}
	Forschungsfrage:
	Wo werden Haptische Schnittstellen bereits heute zur Aufmerksamkeitssteuerung, bei Beobachtugnsaufgaben, eingesetzt und wie könnte man diese Bereiche erweitern?
	
	Um diese Frage beantworten zu können muss ich zuerst:
	\item Klären, was Haptische Schnittstellen sind. Welche Möglichkeiten zur  Entwicklung und Anpassung es gibt. Welche Probleme sie gemeinsam haben
	\item Wie Aufmerksamkeit, zumindest auf einem Abstrakten Niveau, zustande kommt. Wieso eine Beeinflussung durch haptische Schnittstellen sinnvoll ist. ?Welche Probleme auftreten?
	\item Was mit Überwachungstätigkeit gemeint ist. Welche besonderen Aspekte zu berücksichtigen sind.
	\item Anhand dieses Rahmens kann ich dann sinnvolle Bereiche auswählen und zusammenführen.
\end{flushleft}
%Inhaltsverzeichnis
%ToDo: Überdenken, ob nicht entfernen
\newpage
\tableofcontents
\newpage

%Hinführung zu der Arbeit
\newpage
\section{Einleitung}
Fragen/Teilgebiete/Gliederungspunkte/Absätze:
Motivation?
Aufmerksamkeit kann vereinfacht als begrenzte Ressource angesehen werden.
Wovon handelt die Arbeit?
Was ist ihr Ziel?
Welche Erkenntisse sind zu finden?
Wie kann ich zu dem Thema hinführen?

%Beschreibung der Teilgebiete, die im späteren Verlauf zusammengeführt werden
\section{Die Thematischen Teilgebiete}
Fragen/Teilgebiete/Gliederungspunkte/Absätze:
Warum ist die Unterteilung in diese Teilgebiete wichtig?
Diese Arbeit erkundet die Schnittmenge der drei Teilgebiete Aufmerksamkeitssteuerung, haptische Schnittstellen und Beobachtungsaufgaben. Jedes dieser Teilgebiete enthält viele Informationen, die den Rahmen dieser Arbeit sprengen würde.
Wo grenzen sie sich ab?
Was nicht in dieser Arbeit zu finden ist sind die Schnittmengen von nur zwei dieser Gebiete. Es wird keine Aufmerksamkeitssteuerung in einer Beobachtungsaufgabe behandelt, die nicht durch eine haptische Schnittstelle erreicht wird.
Welche Gebiete wären sonst noch wichtig gewesen, werden aber wegen einem zu großen Umfang ausgelassen?
Gerade die Schnittmenge zwischen Beobachtungsaufgabe und Aufmerksamkeitssteuerung ist besonders groß. Hier gibt es zu jedem menschlichen Sinn eine Verwendungsmöglichkeit ihn zur Aufmerksamkeitsgewinnung einzusetzen.
Wie sind diese Unterteilungen zu stande gekommen?

Die Arbeit setzt sich aus drei Teilen zusammen. Dabei ist das erste das Anfällt die Aufmerksamkeitssteuerung. Also wie bringe ich jemanden dazu dort hin zu schauen, wo die Aktion ist. Gefolgt von Haptischen Schnittstellen. Diese sollen im Unterschied zu Ton überwiegend über die Haut Informationen übertragen. Zulätz wird das ganze in den Rahmen einer Überwachungsaufgabe gefasst.

\subsection{Aufmerksamkeitssteuerung}
Aufmerksamkeit ist ein weitläufiges Feld, das es für den Zweck dieser Arbeit zu konkretisieren gilt. Daniel Kahneman \cite{kahneman1973attention} beschreibt unterschiedliche Eigenschaften der Aufmerksamkeit, von denen die Selektierende Eigenschaft die wichtigste, für diese Arbeit, ist.
Die Aufmerksamkeit wird in dieser Arbeit als menschliche Ressource aufgefasst, deren Verteilung es zu steuern gilt. Somit werden beispielsweise die Bereiche "Aspekte der Intensität"\cite{kahneman1973attention} und "Erregung"\cite{kahneman1973attention} ignoriert.

Eine Steuerung wird immer dann erreicht, wenn ein Stimulus verwendet wird, der die Aufmerksamkeit, eines Menschen, zu der gewünschten Information leitet. Diese Aufmerksamkeitssteuerung kann über jeden Sinn erfolgen. Beispiele wären das Ansprechen eines Menschen mit dem Namen und das Einblenden eines Warnsymbols im Auto.

\newpage
\subsection{Überwachungsaufgaben}
Überwachungsaufgaben fordern von dem Aufgaben-Ausführer, dass er über einen Längeren Zeitraum Informationen aufnimmt und überwacht. Überwachen heißt dabei, dass der Aufgaben-Ausführer möglichst schnell auf Veränderungen reagieren kann. Ein Beispiel hierfür wäre ein Sicherheitsbeauftragter, der Überwachungsmonitore überprüft. Auszeichnendes Merkmal, der Aufgabe, ist, dass der Großteil der Zeit der Großteil der Informationen unverändert bleiben.

\newpage
\subsection{Haptische Schnittstellen}
Der Mensch verfügt über einen Tastsinn. Um Informationen über diesen Sinn übertragen zu können, werden haptische Schnittstellen verwendet. 

 %Abschnitt über die Haut
 
Die für den Tastsinn verantwortlichen Nervenzellen können auf unterschiedliche Arten stimuliert werden. Dementsprechen gibt es unterschiedliche haptische \glspl{Aktuator}, die zu Informationsübertragung verwendet werden können. Dabei ist eine Unterscheidung zwischen \glspl{Aktuator} zu treffen. Die Kommunikation kann entweder über mechanische Bewegung oder elektrische Impulse erfolgen. Darüber hinaus lassen sich weitere Charakteristiken erkennen:

Für beiden \glspl{Aktuator}-Typen vergleichbar sind folgende Charakteristiken:
\begin{itemize}
	\item Position auf der Haut
	
	%Die Haut reagiert nicht an jeder Stelle gleich empfindlich auf haptische Stimulation\cite[S.~91]{doi:10.1518/001872008X250638}.
	%Die empfindlichsten Stellen liegen in den Fingerspitzen.
	\item Berührungsfläche
	
	\item Dauer
	%Die Länge der Aktivierung des \gls{Aktuator} beeinflusst 
\end{itemize}

Für die Kommunikation über Vibrationen\cite{doi:10.1518/001872008X250638}:
\begin{itemize}
	\item Frequenz
	\item Amplitude/Intensität %Hier sollte ich mich wahrscheinlich auf eines festlegen
	\item Dauer
%Hier könnte vielleicht ein extra Punkt draus gemacht werden. Alternativ kann ich es als Fazit der Kombination der einzelnen Punkte anführen.
\end{itemize}

Für die Kommunikation über elektrische Impulse\cite[S.~4]{68204}:
\begin{itemize}
	\item Strom
	\item Spannung
	\item Material
	\item Feuchtigkeit
\end{itemize}

In beiden Fällen ist auch die Kombination der einzelnen Faktoren ausschlaggebend, wie effektiv die Kommunikation stattfindet. Dabei stellt jede Ausprägung dieser Kombinationen ein Aktivierungsmuster da. Diese Aktivierungsmuster werden von Menschen nicht nur mit unterschiedlichen Informationen, sondern auch mit subjektiven Emotionen belegt\cite{5444662}.%ToDo Quelle
%Vielleicht nicht hier darstellen, sondern unter eine spezielle Anwendung stellen.

Ein Zusammenschluss von mehreren haptischen \glspl{Aktuator} führt zu einer größeren Anzahl von Einstellungsmöglichkeiten. Diese ermöglichen das übertragen von komplexeren Informationen im Vergleich zu einem haptischen \gls{Aktuator}. Eine Alternative Einsatzmöglichkeit ist zu der Erhöhung der Redundanz bei der Informationsübertragung. Dabei senden die haptischen \glspl{Aktuator}, beispielsweise, alle das selbe Übertragungsmuster. Das zu erreichene Ziel ist hierbei dem Menschen, der haptische \gls{Aktuator} auf der Haut trägt, die Aufnahme der Information zu erleichtern. Diese Anwendung ist gerade in kritischen Situationen,wie sie etwa in militärischen Einsätzen zu finden sind, hilfreich\cite{nikolic1998multisensory}. Nikolic et al. \cite{nikolic1998multisensory} beschreibt, wie haptische Aktoren Piloten bei der Überwachung von Flugzeugdaten unterstützen kann.
Je nach Einsatzbereich können zusätzliche Einschränkungen gelten. In dem Bereits angesprochenen Militärbeispiel ist eine Verwendung von Aktoren, die an dem Finger angebracht sind, nicht sinnvoll. Ein Positionierung an den Fingern würde die Verwendung desselben einschränken.

%Weitere Punkte, die ich integrieren will:
Lösungen oft durch vereinfachung der datenübertragung. Also genaue Signale gegenüber komplexen, vermeindlich informationsreichen.

Auslagerung der komplexen Informationsaufnahme an andere Sinnesorgane. Verwende Haptik nur um Aufmerksamkeit zu gewinnen. So wie evolutionär entwickelt. \cite{doi:10.1518/001872008X250638}

\section{Anwendungen}
Fragen/Teilgebiete/Gliederungspunkte/Absätze:
Nun stellt sich die Frage in welchen Ausprägungen diese Teilgebiete zusammengeführt werden können. Deshalb sollen im folgenden Anwendungen, die alle drei Teilgebiete umfassen beleuchtet werden.
\subsection{Sinneswiederherstellung}
Fragen/Teilgebiete/Gliederungspunkte/Absätze:
Menschliche Sinne können, von Geburt an oder im laufe der Zeit, nicht, oder nur eingeschränkt, funktionsfähig sein. Um diesen Leistungsverlust ausgleichen zu können bedarf es technischer Hilfsmittel. Hierbei bietet die menschliche Haut eine Möglichkeit zur Kommunikation mit der Außenwelt. Im folgenden soll beschrieben werden, wie diese Eigenschaft genutzt werden kann um, über haptische Schnittstellen, ausgewählte Sinne wiederherstellen zu können.

Die Augen stellen eine mächtige Verbindung zur Ausenwelt da. Deshalb ist eine eins zu eins Übersetzung über die Haut nur schwer vorstellbar. Deshalb geht es bei dieser Fragestellung darum die Komplexität der Informationen zu reduzieren. Beispielsweise könnte geschriebene Schrift von einer Kamera erfasst und in eine Brail-artige Schrift übersetzt werden, die unter dem Finger des Anwenders manifestiert wird. Diese direkte Übersetzung bietet eine gute Möglichkeit das Prinzip der Komplexitätsreduktion zu erkennen. Das Problem dabei ist auch, dass die Haut nicht beliebig schnell Unterschiede wahrnehmen kann. Außerdem ist die Interpretation der Signale durch den Menschen ein weiterer Engpass. So wäre es ansonste beispielsweise vorstellbar das Übersetzungsproblem durch 26 haptische Aktoren zu lösen. Dabei würde jeder Aktor zu einem Buchstaben im Alphabet zugeordnet werden. Die einzelnen Aktoren seien entlang des Unterarmes angeordnet. Die Differenzierung der Aktoren ist jetzt jedoch zu anspruchsvoll, wenn sich die Aktivierungsmuster der Aktoren nur durch ihre Position auf dem Körper unterscheiden.

\subsubsection{Sehvermögen}
Fragen/Teilgebiete/Gliederungspunkte/Absätze:
Nach dem Stand der aktuellen Forschung ist das Auge das Leistungsstärkste Sinnesorgan, gemessen an der übertragenen Datenmenge\cite{Koch2006}. Dabei liegt die absolute Leistung ca. bei der eines Ethernet-Kabels mit 10 Mbit/s\cite{Koch2006}. Der Sehsinn kann somit bereits aus technischen gründen nicht vollständig über die Haut simuliert werden.
\subsection{Zwischenmenschliche Kommunikation}
Fragen/Teilgebiete/Gliederungspunkte/Absätze:
Eine der grundlegenden menschlichen Fähigkeiten ist die Kommunikation. Hierbei handelt es sich um den Austausch von komplexen Informationen. Dieser Informationenaustausch soll in diesem Abschnitt ausschließlich von Mensch zu Mensch erfolgen.
Die Traditionelle Kommunikation basiert auf der Stimme und dementsprechen beim Zuhörer auf den Ohren. Im Bereich Sinneswiederherstellung wurde bereits besprochen, wie man diesen traditionellen Kommunikationsweg wiederherstellen kann. In diesem Kapitel soll es daher um die Erzeugung neuartiger Kommunikationswege gehen.

Quellen, die ich verwenden will:
Some Neglected Possibilities of Communication\cite{10.2307/1705360}

\subsection{Leistungssteigerung}
Fragen/Teilgebiete/Gliederungspunkte/Absätze:
Leistung ist nach der Physik Arbeit pro Zeit. Um eine Leistungssteigerung zu erreichen muss also entweder die geleistete Arbeit bei gleicher Zeit erhöht werden oder dementsprechend die Zeit kürzer werden, die für eine Aufgabe gefragt ist.
Hier können haptische Aktuatoren unterstützend eingreifen.


\subsection{Erweiterung des Wahrnehmungsspektrums}
Fragen/Teilgebiete/Gliederungspunkte/Absätze:
Das Wahrnehmungsspektrum des Menschen ist durch die ihm zur Verfügung stehenden Sinne begrenzt. Auch spielt die Verarbeitungsgeschwindigkeit dieser Informationen für die Gesamtwahrnehmung eine Rolle.
Die evolutionäre Aufgabe der haptik ist auf kurzer Distanz (Berührung) Informationen über die Umwelt zu liefern. Dementsprechen ist es von der Natur nicht vorgesehen größere Distanzen haptisch zu erfassen.
Jedoch sind Situationen denkbar, in den eine Verlagerung der Umgebungsanalyse von den Augen, die die Hauptverantwortlichen hierfür sind, auf andere Sinnesorgane vorzunehemen. Dadurch werden andere Sinneskapazitäten freigeräumt.

\paragraph{Haptische Navigationssysteme}

\subsection{Zuverlässigkeit Erzeugung}
Fragen/Teilgebiete/Gliederungspunkte/Absätze:

\section{Zusammenfassung und Ausblick}
Fragen/Teilgebiete/Gliederungspunkte/Absätze:

%Zusätzliche Informationen und Formalitäten
\section{Anhang}

\clearpage
\subsection{Glossar}\label{glossar}

\renewcommand*{\glossarysection}[2][]{}	% prevents double glossary section heading
\printnoidxglossaries				% generate pdf twice when adding new entries

\subsection{Selbständigkeitserklärung}

\clearpage
\bibliography{Literaturreferenzen}

\end{document}
