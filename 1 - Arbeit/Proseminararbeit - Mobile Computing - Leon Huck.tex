% Auf der Basis von dem Springer llncs Style
\documentclass{llncs}					% Springer Style
\usepackage{llncsdoc}					% Springer Dokument
\usepackage[utf8]{inputenc}				% Umlaute, Sonderzeichen
\usepackage[ngerman]{babel}				% deutsche Sprache
\usepackage{enumitem}					% Listen
\usepackage{graphicx}					% Grafiken
\usepackage{hyperref}					% Hyperlinks
\usepackage[nonumberlist]{glossaries}	% Glossar

\bibliographystyle{unsrt}

\makenoidxglossaries

\newglossaryentry{Aktuator}{
	name=Aktuator,
	plural=Aktuatoren,
	description={Bauelement, welches elektrische Signale in andere physikalische Größen, wie beispielsweise Bewegung, umsetzt.}
}

\title{Aufmerksamkeitssteuerung durch Haptische Schnittstellen in Überwachungstätigkeiten}
\author{Leon Huck\thanks{Unter der Betreuung von: Erik Pescara}}
\institute{Karlsruher Institut für Technologie}
\date{17.06.2019}

\begin{document}
	
%Titel der Arbeit
\maketitle

%Schlagworte der Arbeit
\begin{description}
	\item 
\end{description}

%Abstrakt der Arbeit
\begin{abstract}
	
\end{abstract}

%Inhaltsverzeichnis
%ToDo: Überdenken, ob nicht entfernen
\newpage
\tableofcontents
\newpage

%Hinführung zu der Arbeit
\newpage
\section{Einleitung}


Seit der Einführung von Geräten, wie etwa Handies, sind Aufmerksamkeitshinweise durch Vibrationen im Alltag angekommen.
Doch bieten Vibrationen auch außerhalb von einem einfachen Alarmsystem eine Vielzahl an Möglichkeiten zur Informationsübermittlung (vgl. \cite{10.2307/1705360}). In dieser Arbeit soll ein Teilbereich der Haptischen Schnittstellen herausgenommen und der Fortschritt in diesem Feld beschrieben werden.

Um dieses Ziel zu erreichen werden die Definitionen Aufmerksamkeitssteuerung und Überwachungsaufgaben eingeführt um anschließend deren Schnittmenge zu betrachten. Dabei werden die Teilbereiche anhand von konkreten Anwendungen erarbeitet und die allgemeinen Erkenntnisse herausgezogen.
Dadurch soll der Stand der Wissenschaft festgehalten und noch vorhandene Wissenslücken zum Vorschein gebracht werden.

%Beschreibung der Teilgebiete, die im späteren Verlauf zusammengeführt werden
\section{Die Thematischen Teilgebiete}
Diese Arbeit erkundet die Schnittmenge der drei Teilgebiete Aufmerksamkeitssteuerung, haptische Schnittstellen und Beobachtungsaufgaben. Jedes dieser Teilgebiete enthält viele Informationen, die den Rahmen dieser Arbeit sprengen würde. Was nicht in dieser Arbeit zu finden ist sind die Schnittmengen von nur zwei dieser Gebiete. Es wird keine Aufmerksamkeitssteuerung in einer Beobachtungsaufgabe behandelt, die nicht durch eine haptische Schnittstelle erreicht wird. Im Folgenden soll eine Einführung und Abgrenzung der Teilgebiete erfolgen.

\subsection{Aufmerksamkeitssteuerung}
Aufmerksamkeit ist ein weitläufiges Feld, das es für den Zweck dieser Arbeit zu konkretisieren gilt. Daniel Kahneman \cite{kahneman1973attention} beschreibt unterschiedliche Eigenschaften der Aufmerksamkeit, von denen die Selektierende Eigenschaft die Wichtigste, für diese Arbeit, ist.
Die Aufmerksamkeit wird in dieser Arbeit als menschliche Ressource aufgefasst, deren Verteilung es zu steuern gilt. Somit werden beispielsweise die Bereiche ''Aspekte der Intensität''\cite{kahneman1973attention} und ''Erregung''\cite{kahneman1973attention} ignoriert.

Eine Steuerung wird immer dann erreicht, wenn ein Stimulus verwendet wird, der die Aufmerksamkeit, eines Menschen, zu der gewünschten Information leitet. Diese Aufmerksamkeitssteuerung kann über jeden Sinn erfolgen. Beispiele wären das Ansprechen eines Menschen mit dem Namen und das Einblenden eines Warnsymbols im Auto. Vorweggreifend soll hier auch eine Anwendung, wie die Handyvibration nicht unerwähnt bleiben.

\subsection{Überwachungsaufgaben}
Überwachungsaufgaben fordern von dem Aufgaben-Ausführer, dass er über einen längeren Zeitraum Informationen aufnimmt und überwacht. Überwachen heißt dabei, dass der Aufgaben-Ausführer möglichst schnell auf Veränderungen reagieren kann.
Ein Beispiel hierfür wäre ein Sicherheitsbeauftragter, der Überwachungsmonitore überprüft.
Angenommen die Überwachung findet Nachts statt. Auszeichnendes Merkmal der Überwachungsaufgabe ist, in diesem Fall, dass der Großteil der Zeit der Großteil der Informationen unverändert bleibt.
Im Gegensatz dazu steht die Überwachung bei Tag. Hier sind potentiell viele Veränderungen erkennbar, jedoch ist nur ein kleiner Teil für die Überwachungsaufgabe wichtig \cite{doi:10.1177/001872087902100109}.
Dieses Beispiel zeigt, dass eine Differenzierung von Überwachungsaufgaben nötig ist um diese vereinfachen oder ermöglichen zu können.

Als allgemeine Ziele von allen Geräten, die Überwachungsaufgaben unterstützen lassen sich festhalten:
\begin{itemize}
	\item Die Aufmerksamkeit  des Aufgaben-Ausführers soll auf, für die Erfüllung der Aufgabe, relevante Informationen geleitet werden, ohne das es zu einer Ermüdung kommt.
	\item Es soll ermöglicht oder vereinfacht werden die Informationen in relevant und irrelevant zu unterteilen.
\end{itemize}

\subsection{Haptische Schnittstellen}
Der Mensch verfügt über einen Tastsinn. Um Informationen über diesen Sinn übertragen zu können, werden haptische Schnittstellen verwendet. 

 %Abschnitt über die Haut
 
Die für den Tastsinn verantwortlichen Nervenzellen können auf unterschiedliche Arten stimuliert werden. Dementsprechen gibt es unterschiedliche haptische \glspl{Aktuator}, die zu Informationsübertragung verwendet werden können. Dabei ist eine Unterscheidung zwischen \glspl{Aktuator} zu treffen. Die Kommunikation kann entweder über mechanische Bewegung oder elektrische Impulse erfolgen. Darüber hinaus lassen sich weitere Charakteristiken erkennen:

Für beiden \glspl{Aktuator}-Typen vergleichbar sind folgende Charakteristiken:
\begin{itemize}
	\item Position auf der Haut
	
	%Die Haut reagiert nicht an jeder Stelle gleich empfindlich auf haptische Stimulation\cite[S.~91]{doi:10.1518/001872008X250638}.
	%Die empfindlichsten Stellen liegen in den Fingerspitzen.
	\item Berührungsfläche
	
	\item Dauer
	%Die Länge der Aktivierung des \gls{Aktuator} beeinflusst 
\end{itemize}

Für die Kommunikation über Vibrationen\cite{doi:10.1518/001872008X250638}:
\begin{itemize}
	\item Frequenz
	\item Amplitude/Intensität %Hier sollte ich mich wahrscheinlich auf eines festlegen
	\item Dauer
%Hier könnte vielleicht ein extra Punkt draus gemacht werden. Alternativ kann ich es als Fazit der Kombination der einzelnen Punkte anführen.
\end{itemize}

Für die Kommunikation über elektrische Impulse\cite[S.~4]{68204}:
\begin{itemize}
	\item Strom
	\item Spannung
	\item Material
	\item Feuchtigkeit
\end{itemize}

In beiden Fällen ist auch die Kombination der einzelnen Faktoren ausschlaggebend, wie effektiv die Kommunikation stattfindet. Dabei stellt jede Ausprägung dieser Kombinationen ein Aktivierungsmuster da. Diese Aktivierungsmuster werden von Menschen nicht nur mit unterschiedlichen Informationen, sondern auch mit subjektiven Emotionen belegt\cite{5444662}.%ToDo Quelle
%Vielleicht nicht hier darstellen, sondern unter eine spezielle Anwendung stellen.

Ein Zusammenschluss von mehreren haptischen \glspl{Aktuator} führt zu einer größeren Anzahl von Einstellungsmöglichkeiten. Diese ermöglichen das übertragen von komplexeren Informationen im Vergleich zu einem haptischen \gls{Aktuator}. Eine Alternative Einsatzmöglichkeit ist zu der Erhöhung der Redundanz bei der Informationsübertragung. Dabei senden die haptischen \glspl{Aktuator}, beispielsweise, alle das selbe Übertragungsmuster. Das zu erreichene Ziel ist hierbei dem Menschen, der haptische \gls{Aktuator} auf der Haut trägt, die Aufnahme der Information zu erleichtern. Diese Anwendung ist gerade in kritischen Situationen,wie sie etwa in militärischen Einsätzen zu finden sind, hilfreich\cite{nikolic1998multisensory}. Nikolic et al. \cite{nikolic1998multisensory} beschreibt, wie haptische Aktoren Piloten bei der Überwachung von Flugzeugdaten unterstützen kann.
Je nach Einsatzbereich können zusätzliche Einschränkungen gelten. In dem Bereits angesprochenen Militärbeispiel ist eine Verwendung von Aktoren, die an dem Finger angebracht sind, nicht sinnvoll. Ein Positionierung an den Fingern würde die Verwendung desselben einschränken.

\newpage
\section{Anwendungen}
Nun stellt sich die Frage in welchen Ausprägungen diese Teilgebiete zusammengeführt werden können. Deshalb sollen im folgenden Anwendungen, die alle drei Teilgebiete umfassen beleuchtet werden.

\subsection{Sinneswiederherstellung}
Menschliche Sinne können, von Geburt an oder im laufe der Zeit, nicht, oder nur eingeschränkt, funktionsfähig sein. Um diesen Leistungsverlust ausgleichen zu können bedarf es technischer Hilfsmittel. Hierbei bietet die menschliche Haut eine Möglichkeit zur Aufnahme von Informationen, die typischerweise über andere Sinne aufgenommen werden würden.

\subsubsection{Sehvermögen}
Fragen/Teilgebiete/Gliederungspunkte/Absätze:
Nach dem Stand der Forschung ist das Auge das Leistungsstärkste Sinnesorgan, gemessen an der übertragenen Datenmenge\cite{Koch2006}. Dabei liegt die absolute Leistung ca. bei der eines Ethernet-Kabels mit 10 Mbit/s\cite{Koch2006}. Der Sehsinn kann somit bereits aus technischen gründen nicht vollständig über die Haut simuliert werden.
Die für die Überwachung der Umwelt wichtigen Informationen lassen sich von den unwichtigen differenzieren.
\subsubsection{Lesen} Geschriebene Worte sind eine Darstellung der menschlichen Sprache. Im Fall der Einschränkung des Sehvermögens ist auch die Fähigkeit zu lesen beeinträchtigt.

\paragraph{Optacon} Eine Lösung für diese Einschränkung wurde von Bliss et al. 1970 in Form des "Optacon" entwickelt (Zitiert nach:\cite{doi:10.1518/001872008X250638}). Dabei werden auf einer Anzeigefläche die Buchstaben in Form von Vibrationen dargestellt. Das identifiezieren der Buchstaben übernimmt ein Scanner, der über geschrieben Worte bewegt werden kann. Mit diesem Gerät ist war es möglich zwischen 50 und 100 Worte in der Minute zu lesen \cite{doi:10.1518/001872008X250638}.
Bliss et al. \cite{4081931} identifiert in seiner Arbeit drei Tests Charakteristiken, die einen Einblick in die Leistung eines "Direkt Übersetzers mit taktilem Ausgang" \cite{4081931} bieten.

\begin{itemize}
	\item Lesbarkeit\newline Die Lesbarkeit beschreibt, mit welcher Wahrscheinlichkeit, die gelesene Information von dem Benutzer, wie vorgesehen interpretiert wird. Für das Erreichen der Charakteristik muss es möglich sein Buchstaben zu unterscheiden. Auch ist die Erneuerungsrate, mit dem das Gerät die Buchstaben neu zeichnet, von Bedeutung. Eine zu geringe Wiederholungsrate kann zu Missverständnissen führen.
	\item Lesegeschwindigkeit
	\item Lesbarer Ausschnitt
\end{itemize}


\subsection{Zwischenmenschliche Kommunikation}

Die Zwischenmenschliche Kommunikation ist ein komplexer Vorgang, bei dem zumeist viele Sinne beansprucht werden. Über den Hörsinn werden die Informationen aufgenommen, die in der gesprochenen Sprache zu finden sind. Der Sehsinn wird verwendet um Lippen zu lesen und somit ein besseres Verständnis zu erzeugen. Darüber hinaus kann über ihn die emotionale Lage des Gesprächspartners eingeschätzt werden und auf Gesten, wie ein Handschlag, reagiert werden.
Jedoch gibt es auch Umgebungen, in denen diese Kommunikationswege unterbunden werden. Der Geräuschpegel kann zu hoch sein um Sprache zu verstehen. Das Licht kann zu dunkel sein um den anderen Menschen zu sehen, mit dem Kommuniziert wird \cite{10.2307/1705360}.

Des weiteren können auch durch Unfälle, Alter oder Krankheiten die Augen und Ohren beeinträchtigt sein. Um die Fähigkeit der Zwischenmenschlichen Kommunikation zu erhalten sind Seh- und Hörhilfen verbreitete technische Werkzeuge. Eine weitere Alternative ist das Umverlagern der Kommunikation auf einen anderen Sinn\cite{10.2307/1705360}.

Frank A. Geldard \cite{10.2307/1705360} beschreibt hierzu in seiner Arbeit die Entwicklung der Forschung, die versucht die zwischenmenschliche Kommunikation auf den Tastsinn zu verlagern. Zuerst wird beschrieben, wie die Haut dazu genutzt werden kann wie ein Ohr zu funktionieren. Dabei wird die Haut als Trommelfell verwendet. Dieser Ansatz liefert nach einer Einlernphase von 30h ein Vokabular von einigen einzelnen Worten\cite{10.2307/1705360}. Das Problem bei dieser Anwendung liegt in der Zuverlässigkeit. Bei einer Wiederkennungsrate von ca. 75% ist die Wahrscheinlichkeit, dass die gesagten Worte nicht oder falsch verstanden werden hoch\cite{10.2307/1705360}. Eine Lösung für dieses Problem kann das Abändern des Vokabulares sein. Anstelle, dass Buchstaben übertragen werden können auch Vibrationsbilder übermittelt werden\cite{10.2307/1705360}.

\paragraph{Tactons} Die Idee, für die haptische Wahrnehmung spezialisierte Vibrationsmuster zu erstellen, wird von Stephen Brewster und Lorna M. Brown\cite{Brewster:2004:TST:976310.976313} behandelt. Ihr Vorschlag ist sogenannte Tactile Icons (Tactons) zu verwenden, die haptisch gut differenzierbar sind. Dabei orientieren sie sich an musik ähnlichen Mustern, die von den taktilen Aktoren dargestellt werden\cite{Brewster:2004:TST:976310.976313}.
Der Unterschied zu der Darstellung von Buchstaben ist, dass die Tactons selbst eine Bedeutung haben und nicht in eine gesprochene Sprache übersetzt werden müssen um sie zu verstehen. Dadurch sollte eine bessere Antwortzeit bei dem Benutzer erreicht werden.

\subsection{Leistungssteigerung}


\subsection{Erweiterung des Wahrnehmungsspektrums}


\subsection{Zuverlässigkeit Erzeugung}

\newpage
\section{Zusammenfassung und Ausblick}

%Zusätzliche Informationen und Formalitäten
\newpage
\section{Anhang}

\clearpage
\subsection{Glossar}\label{glossar}

\renewcommand*{\glossarysection}[2][]{}	% prevents double glossary section heading
\printnoidxglossaries				% generate pdf twice when adding new entries

\subsection{Selbständigkeitserklärung}

\clearpage
\bibliography{Literaturreferenzen}

\end{document}
