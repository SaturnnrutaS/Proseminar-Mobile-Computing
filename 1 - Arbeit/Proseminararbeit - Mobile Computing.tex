% Auf der Basis von dem Springer llncs Style
\documentclass{llncs}					% Springer Style
\usepackage{llncsdoc}					% Springer Dokument
\usepackage[utf8]{inputenc}				% Umlaute, Sonderzeichen
\usepackage[ngerman]{babel}				% deutsche Sprache
\usepackage{enumitem}					% Listen
\usepackage{graphicx}					% Grafiken
\usepackage{hyperref}					% Hyperlinks
\usepackage[nonumberlist]{glossaries}	% Glossar
\usepackage{amsmath}					% Mathematik

\bibliographystyle{unsrt}

\makenoidxglossaries

\newglossaryentry{Aktuator}{
	name=Aktuator,
	plural=Aktuatoren,
	description={Bauelement, welches elektrische Signale in andere physikalische Größen, wie beispielsweise Bewegung, umsetzt.}
}

\title{Aufmerksamkeitssteuerung durch Haptische Schnittstellen in Überwachungstätigkeiten}
\author{Leon Huck\thanks{Unter der Betreuung von: Erik Pescara}}
\institute{Karlsruher Institut für Technologie}
\date{17.06.2019}

\begin{document}
	
%Titel der Arbeit
\maketitle

%Schlagworte der Arbeit
\begin{description}
	\item ToDo
\end{description}

%Abstrakt der Arbeit
\begin{abstract}
	ToDo
\end{abstract}

\begin{flushleft}
	Forschungsfrage:
	Wo werden Haptische Schnittstellen bereits heute zur Aufmerksamkeitssteuerung, bei Beobachtugnsaufgaben, eingesetzt und wie könnte man diese Bereiche erweitern?
	
	Um diese Frage beantworten zu können muss ich zuerst:
	\item Klären, was Haptische Schnittstellen sind. Welche Möglichkeiten zur  Entwicklung und Anpassung es gibt. Welche Probleme sie gemeinsam haben
	\item Wie Aufmerksamkeit, zumindest auf einem Abstrakten Niveau, zustande kommt. Wieso eine Beeinflussung durch haptische Schnittstellen sinnvoll ist. ?Welche Probleme auftreten?
	\item Was mit Überwachungstätigkeit gemeint ist. Welche besonderen Aspekte zu berücksichtigen sind.
	\item Anhand dieses Rahmens kann ich dann sinnvolle Bereiche auswählen und zusammenführen.
\end{flushleft}
%Inhaltsverzeichnis
%ToDo: Überdenken, ob nicht entfernen
\newpage
\tableofcontents
\newpage

%Hinführung zu der Arbeit
\newpage
\section{Einleitung}
Fragen/Teilgebiete/Gliederungspunkte/Absätze:
Motivation?
Aufmerksamkeit kann vereinfacht als begrenzte Ressource angesehen werden.
Wovon handelt die Arbeit?
Was ist ihr Ziel?
Welche Erkenntisse sind zu finden?
Wie kann ich zu dem Thema hinführen?

%Beschreibung der Teilgebiete, die im späteren Verlauf zusammengeführt werden
\section{Die Thematischen Teilgebiete}
Fragen/Teilgebiete/Gliederungspunkte/Absätze:
Warum ist die Unterteilung in diese Teilgebiete wichtig?
Diese Arbeit erkundet die Schnittmenge der drei Teilgebiete Aufmerksamkeitssteuerung, haptische Schnittstellen und Beobachtungsaufgaben. Jedes dieser Teilgebiete enthält viele Informationen, die den Rahmen dieser Arbeit sprengen würde.
Wo grenzen sie sich ab?
Was nicht in dieser Arbeit zu finden ist sind die Schnittmengen von nur zwei dieser Gebiete. Es wird keine Aufmerksamkeitssteuerung in einer Beobachtungsaufgabe behandelt, die nicht durch eine haptische Schnittstelle erreicht wird.
Welche Gebiete wären sonst noch wichtig gewesen, werden aber wegen einem zu großen Umfang ausgelassen?
Gerade die Schnittmenge zwischen Beobachtungsaufgabe und Aufmerksamkeitssteuerung ist besonders groß. Hier gibt es zu jedem menschlichen Sinn eine Verwendungsmöglichkeit ihn zur Aufmerksamkeitsgewinnung einzusetzen.
Wie sind diese Unterteilungen zu stande gekommen?

Die Arbeit setzt sich aus drei Teilen zusammen. Dabei ist das erste das Anfällt die Aufmerksamkeitssteuerung. Also wie bringe ich jemanden dazu dort hin zu schauen, wo die Aktion ist. Gefolgt von Haptischen Schnittstellen. Diese sollen im Unterschied zu Ton überwiegend über die Haut Informationen übertragen. Zulätz wird das ganze in den Rahmen einer Überwachungsaufgabe gefasst.

\subsection{Aufmerksamkeitssteuerung}
Fragen/Teilgebiete/Gliederungspunkte/Absätze:
Aufmerksamkeit ist ein weitläufiges Feld. Deshalb ist es für die Diskussion in der Arbeit wichtig genau zu definieren, welche Arten der Aufmerksamkeit behandelt werden.

\subsection{Haptische Schnittstellen}
Der Mensch verfügt über einen Tastsinn. Um Informationen über diesen Sinn übertragen zu können, werden haptische Schnittstellen verwendet. 

 %Abschnitt über die Haut
 
Diese Nervenzellen können auf unterschiedliche Arten stimuliert werden. Dementsprechen gibt es unterschiedliche Geräte, die zu Informationsübertragung verwendet werden können. Dabei ist eine Unterscheidung zwischen Taktoren zu treffen, die durch mechanische Bewegung kommunizieren und solche, die mittels elktrische Impulse kommunizieren. Ähnlich, wie es in der Sprache unterschiedliche Charakteristiken der Worte gibt so lassen sich auch bei den Taktoren unterschiedliche Charakteristiken aufzeigen. Dabei sind vor allem die unterschiedlichen Charakteristiken von Bedeutung, die genutzt werden können um über den Tastsinn zu kommunizieren:
Für beiden \glspl{Aktuator}-Typen vergleichbar sind folgende Charakteristiken:
\begin{itemize}
	\item Position auf der Haut
	\item Berührungsfläche
	\item Dauer
\end{itemize}

Für die Kommunikation über Vibrationen\cite{doi:10.1518/001872008X250638}:
\begin{itemize}
	\item Frequenz
	\item Amplitude/Intensität %Hier sollte ich mich wahrscheinlich auf eines festlegen
	\item Berührungsfläche
	\item Dauer
	\item Das verwendete Muster der Stimulation %Hier könnte vielleicht ein extra Punkt draus gemacht werden. Alternativ kann ich es als Fazit der Kombination der einzelnen Punkte anführen.
\end{itemize}

Für die Kommunikation über elektrische Impulse\cite[S.~4]{68204}:
\begin{itemize}
	\item Strom
	\item Spannung
	\item Material
	\item Druck %Der Druck mit der die Diode auf der Stelle lastet. In den Text integrieren.
	\item Position auf der Haut
	\item Dicke
	\item Größe der Diode
	\item Feuchtigkeit
\end{itemize}

In beiden Fällen ist auch die Kombination der einzelnen Faktoren ausschlaggebend, wie effektiv die Kommunikation stattfindet. Dabei stellt jede Ausprägung dieser Kombinationen ein Aktivierungsmuster da. Diese Aktivierungsmuster werden von Menschen nicht nur mit unterschiedlichen Informationen, sondern auch mit unterschiedlichen Emotionen belegt.%ToDo Quelle
%Vielleicht nicht hier darstellen, sondern unter eine spezielle Anwendung stellen.

Ein Zusammenschluss von mehreren taktilen \glspl{Aktuator} führt zu einer größeren Anzahl von Einstellungsmöglichkeiten. Diese ermöglichen das übertragen von komplexeren Information, als was ein taktiler Aktuator alleine erreichen kann. Eine Alternative Einsatzmöglichkeit ist zu der Erhöhung der Redundanz der Informationen. Dabei senden die taktilen Aktuatoren alle das selbe Übertragungsmuster. Das zu erreichene Ziel ist hierbei dem Menschen, der taktilen Aktuator auf der Haut trägt, die Aufnahme der Information zu erleichtern. Diese Anwendung ist gerade in kritischen Situationen,wie sie etwa in militärischen Einsätzen zu finden sind, hilfreich\cite{nikolic1998multisensory}.
\subsection{Überwachungsaufgaben}
Fragen/Teilgebiete/Gliederungspunkte/Absätze:

\section{Anwendungen}
Fragen/Teilgebiete/Gliederungspunkte/Absätze:

\subsection{Sinneswiederherstellung}
Fragen/Teilgebiete/Gliederungspunkte/Absätze:
Menschliche Sinne können, von Geburt an oder im laufe der Zeit, nicht, oder nur eingeschränkt, funktionsfähig sein. Um diesen Leistungsverlust ausgleichen zu können bedarf es technischer Hilfsmittel. Hierbei bietet die menschliche Haut eine Möglichkeit zur Kommunikation mit der Außenwelt. Im folgenden soll beschrieben werden, wie diese Eigenschaft genutzt werden kann um, über haptische Schnittstellen, ausgewählte Sinne wiederherstellen zu können.

\subsubsection{Sehvermögen}
Fragen/Teilgebiete/Gliederungspunkte/Absätze:
Nach dem Stand der aktuellen Forschung ist das Auge das Leistungsstärkste Sinnesorgan, gemessen an der übertragenen Datenmenge\cite{Koch2006}. Dabei liegt die absolute Leistung ca. bei der eines Ethernet-Kabels mit 10 Mbit/s\cite{Koch2006}. Der Sehsinn kann somit bereits aus technischen gründen nicht vollständig über die Haut simuliert werden.
\subsection{Kommunikation}
Fragen/Teilgebiete/Gliederungspunkte/Absätze:

\subsection{Leistungssteigerung}
Fragen/Teilgebiete/Gliederungspunkte/Absätze:

\subsection{Wahrnehmungsspektrum Erweiterung}
Fragen/Teilgebiete/Gliederungspunkte/Absätze:

\subsection{Zuverlässigkeit Erzeugung}
Fragen/Teilgebiete/Gliederungspunkte/Absätze:

\section{Zusammenfassung und Ausblick}
Fragen/Teilgebiete/Gliederungspunkte/Absätze:

%Zusätzliche Informationen und Formalitäten
\section{Anhang}

\clearpage
\subsection{Glossar}\label{glossar}

\renewcommand*{\glossarysection}[2][]{}	% prevents double glossary section heading
\printnoidxglossaries				% generate pdf twice when adding new entries

\subsection{Selbständigkeitserklärung}

\clearpage
\bibliography{Literaturreferenzen}

\end{document}
