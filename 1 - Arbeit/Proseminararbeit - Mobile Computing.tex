% Auf der Basis von dem Springer llncs Style
\documentclass{llncs}					% Springer Style
\usepackage{llncsdoc}					% Springer Dokument
\usepackage[utf8]{inputenc}				% Umlaute, Sonderzeichen
\usepackage[ngerman]{babel}				% deutsche Sprache
\usepackage{enumitem}					% Listen
\usepackage{graphicx}					% Grafiken
\usepackage{hyperref}					% Hyperlinks
\usepackage[nonumberlist]{glossaries}	% Glossar
\usepackage{amsmath}					% Mathematik

\bibliographystyle{unsrt}

\makenoidxglossaries

\newglossaryentry{1}{
	name=A Haptic Back Display for Attentional and Directional Cueing,
	description={Der Raspberry Pi ist ein Einplatinencomputer. In diesem Projekt dient der Raspberry Pi als Hardwareplattform, um Messwerte aus angeschlossenen Sensoren auszulesen}
}

\title{Aufmerksamkeitssteuerung durch Haptische Schnittstellen in Überwachungstätigkeiten}
\author{Leon Huck\thanks{Unter der Betreuung von: Erik Pescara}}
\institute{Karlsruher Institut für Technology}
\date{17.06.2019}

\begin{document}
	
%Titel der Arbeit
\maketitle

%Schlagworte der Arbeit
\begin{description}
	\item ToDo
\end{description}

%Abstrakt der Arbeit
\begin{abstract}
	ToDo
\end{abstract}

%Inhaltsverzeichnis
%ToDo: Überdenken, ob nicht entfernen
\newpage
\tableofcontents
\newpage

%Hinführung zu der Arbeit
\newpage
\section{Einleitung}

%Beschreibung der Teilgebiete, die im späteren Verlauf zusammengeführt werden
\section{Die Teilgebiete}
Die Arbeit setzt sich aus drei Teilen zusammen. Dabei ist das erste das Anfällt die Aufmerksamkeitssteuerung. Also wie bringe ich jemanden dazu dort hin zu schauen, wo die Aktion ist. Gefolgt von Haptischen Schnittstellen. Diese sollen im Unterschied zu Ton überwiegend über die Haut Informationen übertragen. Zulätz wird das ganze in den Rahmen einer Überwachungsaufgabe gefasst.

\subsection{Aufmerksamkeitssteuerung}
Aufmerksamkeit ist ein weitläufiges Feld. Deshalb ist es für die Diskussion in der Arbeit wichtig genau zu definieren, welche Arten der Aufmerksamkeit behandelt werden.
\subsection{Haptische Schnittstellen}
\subsection{Überwachungsaufgaben}

\section{Sinneswiederherstellende Anwendungen}
Menschliche Sinne können, von Geburt an oder im laufe der Zeit, nicht, oder nur eingeschränkt, funktionsfähig sein. Um diesen Leistungsverlust ausgleichen zu können bedarf es technischer Hilfsmittel. Hierbei bietet die menschliche Haut eine Möglichkeit zur Kommunikation mit der Außenwelt. Im folgenden soll beschrieben werden, wie diese Eigenschaft genutzt werden kann um, über haptische Schnittstellen, ausgewählte Sinne wiederherstellen zu können.

\subsection{Sehvermögen}
Nach dem Stand der aktuellen Forschung ist das Auge das Leistungsstärkste Sinnesorgan, gemessen an der übertragenen Datenmenge\cite{Koch2006}. Dabei liegt die absolute Leistung ca. bei der eines Ethernet-Kabels mit 10 Mbit/s\cite{Koch2006}. Der Sehsinn kann somit bereits aus technischen gründen nicht vollständig über die Haut simuliert werden.
\section{Kommunikations Anwendungen}

\section{Leistungssteigernde Anwendungen}

\section{Wahrnehmungsspektrum Erweiternde Anwendungen}

\section{Zuverlässigkeit Erzeugende Anwendungen}

\section{Zusammenfassung und Ausblick}

%Zusätzliche Informationen und Formalitäten
\section{Anhang}

\clearpage
\subsection{Glossar}\label{glossar}

\renewcommand*{\glossarysection}[2][]{}	% prevents double glossary section heading
\printnoidxglossaries				% generate pdf twice when adding new entries

\subsection{Selbständigkeitserklärung}

\clearpage
\bibliography{Literaturreferenzen}

\end{document}
