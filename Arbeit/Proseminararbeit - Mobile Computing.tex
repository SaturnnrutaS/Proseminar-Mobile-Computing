% This is LLNCS.DOC the documentation file of
% the LaTeX2e class from Springer-Verlag
% for Lecture Notes in Computer Science, version 2.4
\documentclass{llncs}
\usepackage{llncsdoc}
\usepackage[utf8]{inputenc}			% Umlaute, Sonderzeichen
\usepackage[ngerman]{babel}			% deutsche Sprache
\usepackage{enumitem}				% Listen
\usepackage{graphicx}				% Grafiken
\usepackage{hyperref}				% Hyperlinks
\usepackage[nonumberlist]{glossaries}		% Glossar
\usepackage{amsmath}
%
\begin{document}
\markboth{Attention Guidancen with Haptic Interfaces in Monitoring Tasks}{Attention Guidancen with Haptic Interfaces in Monitoring Tasks}
\thispagestyle{empty}

\begin{flushleft}
\LARGE\bfseries Mobile Computing\\
\end{flushleft}

\rule{\textwidth}{1pt}
\vspace{2pt}

\begin{flushright}
\Huge
\begin{tabular}{@{}l}
Attention Guidancen with\\
Haptic Interfaces in\\
Monitoring Tasks\\
{\Large Version 0.0}
\end{tabular}
\end{flushright}

\rule{\textwidth}{1pt}
\vfill

\newpage
\tableofcontents
\newpage

\newpage
\section{Einleitung}
\section{Die Teilgebiete}
\subsection{Aufmerksamkeitssteuerung}
\subsection{Haptische Schnittstellen}
\subsection{Überwachungsaufgaben}

\section{Wiederherstellung und Ersetzen von Fähigkeiten}
\subsection{Sehvermögen}
\subsection{Hörvermögen}
\subsection{Sprachvermögen}
\section{Nutzung zur Leistungssteigernd}
\subsection{Erlernen von motorsichen Fägikeiten}
\subsection{Erhöhung der Reaktionsgeschwindigkeit}
\section{Erweiterung des Wahrnehmungsspektrums}
\subsection{Navigationssysteme}
\subsection{Nähewahrnehmung}

\clearpage
\section{Glossar}\label{glossar}

\renewcommand*{\glossarysection}[2][]{}	% prevents double glossary section heading
\printnoidxglossaries				% generate pdf twice when adding new entries

\end{document}
