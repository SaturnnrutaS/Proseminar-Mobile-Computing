% Auf der Basis von dem Springer llncs Style
\documentclass{llncs}					% Springer Style
\usepackage{llncsdoc}					% Springer Dokument
\usepackage[utf8]{inputenc}				% Umlaute, Sonderzeichen
\usepackage[ngerman]{babel}				% deutsche Sprache
\usepackage{enumitem}					% Listen
\usepackage{graphicx}					% Grafiken
\usepackage{hyperref}					% Hyperlinks
\usepackage[nonumberlist]{glossaries}	% Glossar
\usepackage{amsmath}					% Mathematik

\bibliographystyle{unsrt}

\makenoidxglossaries

\newglossaryentry{1}{
	name=A Haptic Back Display for Attentional and Directional Cueing,
	description={Der Raspberry Pi ist ein Einplatinencomputer. In diesem Projekt dient der Raspberry Pi als Hardwareplattform, um Messwerte aus angeschlossenen Sensoren auszulesen}
}

\title{Aufmerksamkeitssteuerung durch Haptische Schnittstellen in Überwachungstätigkeiten}
\author{Leon Huck}
\institute{Karlsruher Institut für Technology}
\date{17.06.2019}

\begin{document}
	
%Titel der Arbeit
\maketitle

%Abstrakt der Arbeit
\begin{abstract}
	ToDo
\end{abstract}

%Inhaltsverzeichnis
%ToDo: Überdenken, ob nicht entfernen
\newpage
\tableofcontents
\newpage

\newpage
\section{Einleitung}
\section{Die Teilgebiete}
\subsection{Aufmerksamkeitssteuerung}
\subsubsection{Begriffserklärung und Abgrenzung}
\cite{tan2003haptic}\footnote{ToDo}

\subsection{Haptische Schnittstellen}
\subsection{Überwachungsaufgaben}

\section{Zusammenführung der Teilgebiete}
\subsection{Nutzung zum Ersetzen von Fähigkeiten}
\subsubsection{Sehvermögen}
\subsubsection{Hörvermögen}
\subsubsection{Sprachvermögen}
\subsection{Nutzung zur Leistungssteigernd}
\subsubsection{Erlernen von motorsichen Fägikeiten}
\subsubsection{Erhöhung der Reaktionsgeschwindigkeit}
\subsection{Nutzung zur Erweiterung des Wahrnehmungsspektrums}
\subsubsection{Navigationssysteme}
\subsubsection{Nähewahrnehmung}

\clearpage
\section{Glossar}\label{glossar}

\renewcommand*{\glossarysection}[2][]{}	% prevents double glossary section heading
\printnoidxglossaries				% generate pdf twice when adding new entries

\clearpage
\bibliography{Literaturreferenzen}

\end{document}
